% Mathematical macros useful for Eletrical and Computer Engineering.
\documentclass{article}
\usepackage{etoolbox}
\usepackage{hyperref}
\usepackage{siunitx}

\usepackage{amsmath}
\usepackage{amsthm}
\usepackage{amssymb}
\usepackage{bm}
\usepackage{bbm}
\usepackage{mathtools}
\mathtoolsset{centercolon} % Vertically centered := and =:

% for making index
%\usepackage{makeidx}
%\makeindex

\usepackage{xcolor}

\hypersetup{
pdfauthor={Alick Zhao},
pdfsubject={math},
pdfkeywords={},
unicode=true,
}

\begin{document}

\newcommand{\thetitle }{Mathematical Macros Useful for Eletrical and Computer Engineering}
\hypersetup{pdftitle={\thetitle}}

\title{\bfseries \thetitle}
\author{Alick Zhao}
\date{\small \today}
\maketitle

%\tableofcontents

%\[
%  \lim_{x \to a} \frac{f(x) - f(a)}{x - a}
%\]

\section{Elementary}

% cf. `texdoc mathtools`
\DeclarePairedDelimiter\abs{\lvert}{\rvert}

Absolute values:

\[
\abs*{\frac{a}{b}}
\]

\[
\abs[\Bigg]{\frac{a}{b}}
\]

\newcommand*{\e}{\ensuremath{\mathrm{e}}}
\newcommand*{\ii}{\ensuremath{\mathrm{i}}}
\newcommand*{\jj}{\ensuremath{\mathrm{j}}}
Euler formula:
\[
\e^{\ii \pi} + 1 = 0
\]

Logarithm:
\[
P_\text{\si{dB}} = 10 \log_{10} P
\]
\[
\ln (1+x) \approx x
\]
\[
C = W \log_2 ( 1+ \text{SINR})
\]

\section{Sets}

% cf. `texdoc mathtools`
% just to make sure it exists
\providecommand\given{}
% can be useful to refer to this outside \Set
\newcommand\SetSymbol[1][]{%
\nonscript\:#1\vert
\allowbreak
\nonscript\:
\mathopen{}}
\DeclarePairedDelimiterX\Set[1]\{\}{%
\renewcommand\given{\SetSymbol[\delimsize]}
#1
}

\[
S = \Set{1, 2, 3, \dotsc, 10}.
\]

\[
\Set*{ x \in X \given \frac{\sqrt{x}}{x^2+1} > 1 }
\]

\renewcommand*{\N}{\mathbb{N}}
\newcommand*{\R}{\mathbb{R}}
\newcommand*{\Z}{\mathbb{Z}}

\renewcommand*{\emptyset}{\varnothing}

Special sets:
$\N, \Z, \R$, $\emptyset$.

% Require package: etoolbox
\DeclarePairedDelimiterX\card[1]\lvert\rvert{
\ifblank{#1}{\:\cdot\:}{#1}
}

Cardinality: $\card{}$
\[
\card{S} = 10
\]

\newcommand*{\union}{\cup}
\newcommand*{\intersect}{\cap}

Set operations: $A\intersect B$, $A \union B$

\section{Calculus}

% cf. https://www.tug.org/TUGboat/Articles/tb18-1/tb54becc.pdf
\makeatletter
\newcommand*{\diff}%
  {\@ifnextchar^{\DIfF}{\DIfF^{}}}
\makeatother
\def\DIfF^#1{%
  \mathop{\mathrm{\mathstrut d}}%
  \nolimits^{#1}\gobblespace}
\def\gobblespace{%
  \futurelet\diffarg\opspace}
\def\opspace{%
  \let\DiffSpace\!%
  \ifx\diffarg(%
    \let\DiffSpace\relax
  \else
    \ifx\diffarg[%
      \let\DiffSpace\relax
    \else
      \ifx\diffarg\{%
        \let\DiffSpace\relax
      \fi%
    \fi%
  \fi%
  \DiffSpace}
\makeatother
\newcommand*{\deriv}[3][]{%
  \frac{\diff^{#1}#2}{\diff #3^{#1}}}
\newcommand*{\pderiv}[3][]{%
  \frac{\partial^{#1}#2}%
  {\partial #3^{#1}}}

\[
a\deriv[2]{y}{x} + b \deriv{y}{x} + c = f(x)
\]

\[
\deriv{(\log y)}{x} = \frac{1}{y}\deriv{y}{x}
\]

\[
\diff{z(x,y)} = \pderiv{z}{x}\diff x + \pderiv{z}{y}\diff y
\]

\[
\int_{0}^{\infty} \e^{-t} \diff t
\]

\section{Probability}

\renewcommand*{\P}{\mathbb{P}\probarg}
\DeclarePairedDelimiterX{\probarg}[1]{\{}{\}}{%
  \ifnum\currentgrouptype=16 \else\begingroup\fi
  \activatebar#1
  \ifnum\currentgrouptype=16 \else\endgroup\fi
}
% Based on http://tex.stackexchange.com/a/187363/24521
\newcommand*{\E}{\operatorname{E}\expectarg}
\DeclarePairedDelimiterX{\expectarg}[1]{[}{]}{%
  \ifnum\currentgrouptype=16 \else\begingroup\fi
  \activatebar#1
  \ifnum\currentgrouptype=16 \else\endgroup\fi
}

\newcommand{\innermid}{\nonscript\;\delimsize\vert\nonscript\;}
\newcommand{\activatebar}{%
  \begingroup\lccode`\~=`\|
  \lowercase{\endgroup\let~}\innermid 
  \mathcode`|=\string"8000
}

Probabilities in the \LaTeXe\ way: $\Pr\{X > 1\}$ v.s.
\verb|mathbb| way: $\P{X>1}$.

Expectations: $\E{X}$, $\E{X|Y}$, $\E[\big]{X|Y}$, and
\[
\E*{\frac{1}{2} X | Y}
\]

\newcommand*{\Ind}[1]{\mathbbm{1}\{#1\}}

\[
\Ind{\text{Something happens}}
\]

\section{Vectors}

\let\arrvec\vec
\renewcommand*{\vec}{\bm}

Vector with an arrow $\arrvec{a}$ v.s. in bold $\vec{a}$.

Dot (Inner) products:

\DeclarePairedDelimiterX\dotprod[2]{\langle}{\rangle}{#1,#2}

\[
\dotprod*{A}{\frac{1}{2}}
\]

% Require package: etoolbox
\DeclarePairedDelimiterX\norm[1]\lVert\rVert{
\ifblank{#1}{\:\cdot\:}{#1}
}

Norms: $\norm{}$, $\norm{\vec{a}}$, $\norm{w}_{\vec{g}}^2$

\section{Bold Math}

{\boldmath
$$ X \rightarrow \mathrm{X} \rightarrow \mathbf{X} $$
}

\section{Definitions}
\[
a := b
\]

\appendix

%\printindex

%\bibliographystyle{thubib}
%\def\thudot{}
%\bibliography{myrefs}

\end{document}
%%% Local Variables:
%%% coding: utf-8
%%% mode: latex
%%% TeX-master: t
%%% End:
%%% vim: set sw=2 isk+=\: et tw=70 cc=+1 formatoptions+=mB:
