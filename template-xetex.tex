%# -*- coding: utf-8 -*-
\documentclass[UTF8,a4paper,fontset=none]{ctexart}
\usepackage{fontspec}

\setCJKmainfont
  [BoldFont =SourceHanSansCN-Regular]{SourceHanSerifCN-Regular}
\setCJKsansfont{SourceHanSansCN-Regular}
\setCJKfamilyfont { zhsong }
   {SourceHanSerifCN-Regular}
\setCJKfamilyfont { zhhei }
   {SourceHanSansCN-Regular}
\NewDocumentCommand \songti   { } { \CJKfamily { zhsong } }
\NewDocumentCommand \heiti    { } { \CJKfamily { zhhei } }

\usepackage{hyperref}

\usepackage{graphicx} % 提供包含图片的支持
% useful when you have a separate fig dir
%\graphicspath{{figdir/}}
\usepackage{siunitx} % SI 国际单位制

% bib related
%\usepackage{natbib}

% for making index
%\usepackage{makeidx}
%\makeindex

\usepackage{xcolor} % 使用颜色宏包
% for typesetting source code
\usepackage{listings}
\lstset{
language=,
% several common used languages: C,C++,Matlab...
tabsize=8,
upquote=true,
breaklines=true,
showspaces=false,
showstringspaces=false,
basicstyle=\ttfamily,
keywordstyle=\color[rgb]{0,0,1},
commentstyle=\color[rgb]{0.133,0.545,0.133},
stringstyle=\color[rgb]{0.627,0.126,0.941},
rulesepcolor=\color{red!20!green!20!blue!20},
%numbers=left,
%numberstyle=\tiny,
%frame=shadowbox,
%escapeinside=``
}

% 用于放置很宽的图片
%\newenvironment{narrow}[2]{%
%\begin{list}{}{%
%\setlength{\topsep}{0pt}%
%\setlength{\leftmargin}{#1}%
%\setlength{\rightmargin}{#2}%
%\setlength{\listparindent}{\parindent}%
%\setlength{\itemindent}{\parindent}%
%\setlength{\parsep}{\parskip}}%
%\item[]}{\end{list}}

\hypersetup{
pdfauthor={PDFAUTHOR},
pdfsubject={report},
pdfkeywords={},
unicode=true,
}

% 标点压缩规则
\punctstyle{kaiming}

\begin{document}

\newcommand{\thetitle }{<+标题+>}
\hypersetup{pdftitle={\thetitle}}

\title{\bfseries \thetitle}
\author{AUTHOR}
\date{\small \today}
\maketitle

%\tableofcontents

\section{Basic}
<+正文+>

\index{rule}xx\rule[0.5ex]{1in}{1pt}xx: 1 inch

xx\rule{1cm}{1pt}xx: 1 cm

\section{math}
%\[
%  \lim_{x \to a} \frac{f(x) - f(a)}{x - a}
%\]

{\boldmath
$$ X \rightarrow \mathrm{X} \rightarrow \mathbf{X} $$
}

\section{中文}

%\TeX 必读书目:\cite{tex}。

中文排版,中English混排。FOO\texttrademark
\index{中文}

%{\fangsong 中文排版,中English混排。}
%\CJKunderline{中文排版,中English混排。}
%\CJKunderwave{中文排版,中English混排。}

\section{源程序}

行内代码 \lstinline[language=Python]|str = 'inline'| 也是可以的。

% A float listing example
%\begin{lstlisting}[float,caption=A floating example,language=Pascal]
%for i:=maxint to 0 do
%begin
%{ do nothing }
%end;
%Write('Case insensitive ');
%WritE('Pascal keywords.');
%\end{lstlisting}

%%\lstinputlisting[language={Matlab}]{plot.m}

\appendix

%\printindex

%\bibliographystyle{thubib}
%\def\thudot{}
%\bibliography{myrefs}

%\listoffigures

\end{document}

%%% Local Variables:
%%% coding: utf-8
%%% mode: latex
%%% TeX-master: t
%%% End:
%%% vim: set sw=2 isk+=\: et tw=70 cc=+1 formatoptions+=mB:
