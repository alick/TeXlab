\documentclass[CJKchecksingle]{beamer}
% pass CJKchecksingle to xeCJK
% Author: alick<alick9188@gmail.com>
% 带中文支持的、使用 Beamer 的幻灯片模板
% 使用 xelatex 编译

% This file is modified from a solution template for:

% - Giving a talk on some subject.
% - The talk is between 15min and 45min long.
% - Style is ornate.

% Copyright 2004 by Till Tantau <tantau@users.sourceforge.net>.
%
% In principle, this file can be redistributed and/or modified under
% the terms of the GNU Public License, version 2.
%
% However, this file is supposed to be a template to be modified
% for your own needs. For this reason, if you use this file as a
% template and not specifically distribute it as part of a another
% package/program, I grant the extra permission to freely copy and
% modify this file as you see fit and even to delete this copyright
% notice.

\mode<presentation>
{
\usetheme[secheader]{Boadilla}
\setbeamercovered{transparent}
}

\usepackage{graphicx} % includegraphics support
%\graphicspath{{fig/}} % directories that hold graphics
\usepackage{listings} % for typesetting source code
\usepackage{amsmath} % for math
\usepackage{siunitx} % for si units

\usepackage{fontspec}
\usepackage[UTF8]{ctex}

% xeCJK conf setup
\punctstyle{kaiming}
\renewcommand\CJKfamilydefault{\CJKsfdefault} % for slides

\title
{我的新成果}

\author[short] % (optional, use only with lots of authors)
{AUTHOR}

\institute[short] % (optional, but mostly needed)
{
Department of FOO\\
BAR University
}
% - Use the \inst command only if there are several affiliations.
% - Keep it simple, no one is interested in your street address.

\date % (optional)
{\today}

\subject{Talks}

% Delete this, if you do not want the table of contents to pop up at
% the beginning of each subsection:
%\AtBeginSubsection[]
%{
%  \begin{frame}<beamer>{Outline}
%    \tableofcontents[currentsection,currentsubsection]
%  \end{frame}
%}


% If you wish to uncover everything in a step-wise fashion, uncomment
% the following command:

%\beamerdefaultoverlayspecification{<+->}

% listings setup
\lstset{basicstyle=\ttfamily,breaklines=true}
% hyperref setup
\hypersetup{
%pdfpagemode=FullScreen,
}

\begin{document}

\begin{frame}
\titlepage
\end{frame}

\begin{frame}{Outline}
\tableofcontents
% You might wish to add the option [pausesections]
\end{frame}


% Since this a solution template for a generic talk, very little can
% be said about how it should be structured. However, the talk length
% of between 15min and 45min and the theme suggest that you stick to
% the following rules:

% - Exactly two or three sections (other than the summary).
% - At *most* three subsections per section.
% - Talk about 30s to 2min per frame. So there should be between about
%   15 and 30 frames, all told.

\section{背景}

\begin{frame}{Picture}
%  \begin{figure}[h]
%    \centering
%    \includegraphics[scale=0.4]{foo.png}
%    \caption{}
%  \end{figure}
\end{frame}

%\begin{frame}{Why \TeX\ (and why not)?}
%  \begin{columns}[t]
%    \begin{column}{.45\textwidth}
%      \begin{block}{Pros}
%        \begin{itemize}
%          \item allow focusing on the content without bothering the
%            layout(including TOC, references, etc)
%          \item excel in typesetting math formular
%          \item enhanced in many aspects
%        \end{itemize}
%      \end{block}
%    \end{column}
%
%    \begin{column}{.45\textwidth}
%      \begin{block}{Cons}
%        \begin{itemize}
%          \item not WYSIWYG
%          \item not easy to be \TeX{}pert
%            \bigskip
%          \item only .doc permitted
%        \end{itemize}
%      \end{block}
%    \end{column}
%
%  \end{columns}
%\end{frame}

\section{系统建模}
\subsection{first}
\begin{frame}{FRAME TITLE}
\end{frame}
\subsection{second}


\section*{总结}

\begin{frame}{Summary}
\end{frame}

%\appendix
%\section<presentation>*{\appendixname}
%\subsection<presentation>*{References}
%
%\begin{frame}[allowframebreaks]
%  \frametitle<presentation>{References}
%
%  \bibliographystyle{thubib}
%  \def\thudot{}
%  \bibliography{refs}
%
%\end{frame}

\end{document}
%%% vim: set sw=2 isk+=\: et tw=70 formatoptions+=mB:
